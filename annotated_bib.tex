% Options for packages loaded elsewhere
\PassOptionsToPackage{unicode}{hyperref}
\PassOptionsToPackage{hyphens}{url}
%
\documentclass[
  stu]{apa7}
%
\PassOptionsToPackage{style=apa}{biblatex}
%\PassOptionsToPackage{indent=7ex}{parskip}
%\usepackage{indentfirst}
%\usepackage[margin=1in]{geometry}
%% right side page number
%\usepackage{fancyhdr}
%\fancypagestyle{plain}{
%\fancyhf{}
%\renewcommand{\headrulewidth}{0pt}
%\fancyhead[R]{\thepage}
%}
%\pagestyle{fancy}
%\fancyhf{}
%\renewcommand{\headrulewidth}{0pt}
%\fancyhead[R]{\thepage}
%% double spacing
%\usepackage{setspace}
%\doublespacing
\affiliation{Brian Lamb School of Communication}
\course{COM 212: Interpersonal communication}
\professor{Dr. Hayden Barber}
\duedate{March 31, 2022}
%
\title{Annotated Bibliography}
\author{Sean F. O'Donovan}
\date{}

\usepackage{amsmath,amssymb}
\usepackage{lmodern}
\usepackage{iftex}
\ifPDFTeX
  \usepackage[T1]{fontenc}
  \usepackage[utf8]{inputenc}
  \usepackage{textcomp} % provide euro and other symbols
\else % if luatex or xetex
  \usepackage{unicode-math}
  \defaultfontfeatures{Scale=MatchLowercase}
  \defaultfontfeatures[\rmfamily]{Ligatures=TeX,Scale=1}
\fi
% Use upquote if available, for straight quotes in verbatim environments
\IfFileExists{upquote.sty}{\usepackage{upquote}}{}
\IfFileExists{microtype.sty}{% use microtype if available
  \usepackage[]{microtype}
  \UseMicrotypeSet[protrusion]{basicmath} % disable protrusion for tt fonts
}{}
\makeatletter
\@ifundefined{KOMAClassName}{% if non-KOMA class
  \IfFileExists{parskip.sty}{%
    \usepackage{parskip}
  }{% else
    \setlength{\parindent}{0pt}
    \setlength{\parskip}{6pt plus 2pt minus 1pt}}
}{% if KOMA class
  \KOMAoptions{parskip=half}}
\makeatother
\usepackage{xcolor}
\IfFileExists{xurl.sty}{\usepackage{xurl}}{} % add URL line breaks if available
\IfFileExists{bookmark.sty}{\usepackage{bookmark}}{\usepackage{hyperref}}
\hypersetup{
  pdftitle={Annotated Bibliography},
  pdfauthor={Sean F. O'Donovan},
  hidelinks,
  pdfcreator={LaTeX via pandoc}}
\urlstyle{same} % disable monospaced font for URLs
\setlength{\emergencystretch}{3em} % prevent overfull lines
\providecommand{\tightlist}{%
  \setlength{\itemsep}{0pt}\setlength{\parskip}{0pt}}
\setcounter{secnumdepth}{-\maxdimen} % remove section numbering
\ifLuaTeX
  \usepackage{selnolig}  % disable illegal ligatures
\fi
\usepackage[]{biblatex}
\addbibresource{comPaper1References.bib}

\begin{document}
\maketitle

\noindent \hangindent=7ex \fullcite{walther95}

This is a landmark study from 1995 that establishes that chronemics
matters in Computer Mediated Communication (CMC). Previously, it was
assumed that CMC was devoid of any nonverbal cues, the
``cues-filtered-out'' view. This is one of the first studies to notice
that there are nonverbal, chronemic cues in CMC. It finds that altering
time and date stamps changes the way external observers see the
communication. \textcite{walther95} validates two hypotheses and has
complex results for others. First, it proves (their H2) that a social
message sent at night signals less dominance than a social message sent
during the day and that a task message sent at night is more dominant
than a task message sent during the day. This suggests that even in cmc,
``different parts of the day correspond to different activity contexts,
and formal `business hours' versus informal `after hours' carry
different expectations {[}30{]}'' \autocite[ 361]{walther95}. Second, it
proves (their H3) that a slow reply to a social message indicates
greater affection than a fast reply, but a slow reply to a task message
indicates less affection than a fast reply \autocite[ 370]{walther95}.
Not all of their results yielded simple results.

They had two other hypotheses, which weren't simply proven or disproven,
but had complex interactions. The first, their H1, expects that a
nighttime social message would be more affectionate than a daytime
social message, and that a daytime task message would be more
affectionate than a nighttime task message. The task message is indeed
recieved as more affectionate or positive when sent in daytime
\autocite[ 368]{walther95}, but the social message's perception is
interfered with by the reply speed. The researchers didn't expect the
reply to affect perception of the first message sent, but when someone
replied quickly to a social message sent at night, the first social
message (not the reply) is seen as less affectionate. The second complex
finding (on their H4) is that a faster response suggest a more equal
footing between participants, rather than suggesting that the replier is
less dominant than the sender \autocite[ 371]{walther95}.

\noindent \hangindent=7ex \fullcite{kalman11}

\textcite{kalman11} cites \textcite{walther95} and describes cmc message
latency in terms of expectation violation theory. \textcite{kalman11}
set up an experiment where managers ranked job applicants based on
``vignettes'' (p.~1) which included chronemic information.
\textcite{kalman11} found a complex interaction. For low valued
applicants, the latency of the message did not change much about the
managers' assessment of them, but for high valued applicants, a high
latency was percieved negatively and changed the managers' appraisal of
their application.

That lines up with Walther's claims that task focused messages with high
latencies were perceived negatively. This study also discusses that
cues-filtered-out theory doesn't account for this effect, as
\textcite{walther95} explains.

If there's anything problematic about this study, it is probably that it
isn't very ambitious. \textcite{kalman11} had hypotheses:

\begin{quote}
Hypothesis 1: An e-mail response latency of 1 day will be more expected
and will lead to more positive evaluations of the responder than the
longer response latency of 2 weeks or no response at all.
\end{quote}

\begin{quote}
Hypothesis 2: The effect of e-mail response latency on perceptions of
the responder will be moderated by candidate reward valence. (p.~59)
\end{quote}

\noindent These aren't exactly exciting, but this is the kind of
groundwork that's needed to get the exciting results of
\textcite{kalman13} below.

\noindent \hangindent=7ex \fullcite{kalman13}

This study evaluates a measure called interpost pause. It builds on the
findings of \textcite{kalman11} and SIP theory by using chronemics as
nonverbal cues that participants use in place of normal face to face
cues. \textcite{kalman13} found that people who were more extraverted
``exhibited shorter interpost pauses,'' and that pairs who trusted each
other less had longer interpost pauses (p.~16). Interestingly, the
correlation with trust was stronger than the correlation with
extraversion.

Unfortunately, it's hard to know why these two results exist. For trust,
\textcite{kalman13} suggests it may come down to the assumption that
lying is harder than telling the truth (and therefore should take
longer), but it may also be that people just prefer their conversation
partner to respond quickly and the dislike of slow responses bleeds into
their trust for the person. For extraversion, \textcite{kalman13}
suggests this is mostly because extraverted people talk more and with
less hesitation than intraverted people. Essentially, the same thing
happens in face to face communication because that's just the way people
are.

\printbibliography

\end{document}
