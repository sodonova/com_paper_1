% Options for packages loaded elsewhere
\PassOptionsToPackage{unicode}{hyperref}
\PassOptionsToPackage{hyphens}{url}
%
\documentclass[
]{article}
\title{Annotated Bibliography}
\author{}
\date{\vspace{-2.5em}}

\usepackage{amsmath,amssymb}
\usepackage{lmodern}
\usepackage{iftex}
\ifPDFTeX
  \usepackage[T1]{fontenc}
  \usepackage[utf8]{inputenc}
  \usepackage{textcomp} % provide euro and other symbols
\else % if luatex or xetex
  \usepackage{unicode-math}
  \defaultfontfeatures{Scale=MatchLowercase}
  \defaultfontfeatures[\rmfamily]{Ligatures=TeX,Scale=1}
\fi
% Use upquote if available, for straight quotes in verbatim environments
\IfFileExists{upquote.sty}{\usepackage{upquote}}{}
\IfFileExists{microtype.sty}{% use microtype if available
  \usepackage[]{microtype}
  \UseMicrotypeSet[protrusion]{basicmath} % disable protrusion for tt fonts
}{}
\makeatletter
\@ifundefined{KOMAClassName}{% if non-KOMA class
  \IfFileExists{parskip.sty}{%
    \usepackage{parskip}
  }{% else
    \setlength{\parindent}{0pt}
    \setlength{\parskip}{6pt plus 2pt minus 1pt}}
}{% if KOMA class
  \KOMAoptions{parskip=half}}
\makeatother
\usepackage{xcolor}
\IfFileExists{xurl.sty}{\usepackage{xurl}}{} % add URL line breaks if available
\IfFileExists{bookmark.sty}{\usepackage{bookmark}}{\usepackage{hyperref}}
\hypersetup{
  pdftitle={Annotated Bibliography},
  hidelinks,
  pdfcreator={LaTeX via pandoc}}
\urlstyle{same} % disable monospaced font for URLs
\usepackage[margin=1in]{geometry}
\usepackage{graphicx}
\makeatletter
\def\maxwidth{\ifdim\Gin@nat@width>\linewidth\linewidth\else\Gin@nat@width\fi}
\def\maxheight{\ifdim\Gin@nat@height>\textheight\textheight\else\Gin@nat@height\fi}
\makeatother
% Scale images if necessary, so that they will not overflow the page
% margins by default, and it is still possible to overwrite the defaults
% using explicit options in \includegraphics[width, height, ...]{}
\setkeys{Gin}{width=\maxwidth,height=\maxheight,keepaspectratio}
% Set default figure placement to htbp
\makeatletter
\def\fps@figure{htbp}
\makeatother
\setlength{\emergencystretch}{3em} % prevent overfull lines
\providecommand{\tightlist}{%
  \setlength{\itemsep}{0pt}\setlength{\parskip}{0pt}}
\setcounter{secnumdepth}{-\maxdimen} % remove section numbering
\PassOptionsToPackage{style=apa}{biblatex}
\ifLuaTeX
  \usepackage{selnolig}  % disable illegal ligatures
\fi
\usepackage[]{biblatex}
\addbibresource{comPaper1References.bib}

\begin{document}
\maketitle

\hypertarget{chronemics}{%
\subsection{Chronemics}\label{chronemics}}

One problem I'm running into is that there isn't very much research on
this. The research that is out there specifies work/team communication
or dating relationships; online dating has many papers.

What I want to study is more generally what the nonverbal cues used in
today's ``CMC'' or ``IM'' are. Chronemics is really interesting, but I'm
not sure if there's enough there for 5-7 pages. Some of the larger
questions I'm interested in are:

\begin{itemize}
\tightlist
\item
  Can software developers mine data about quality of conversation from
  simple timestamps?
\item
  Do romantic partners have different norms for speed of response?
\item
  Is there some kind of pacing effect based on word count of response?
\item
  How long does a normal person take to respond to an email? a text?
\item
  Is there an effect when technology nags you to respond?
\end{itemize}

\textbf{\fullcite{walther95}}

This is a landmark study from 1995 that establishes that chronemics
matters in Computer Mediated Communication (CMC). Previously, it was
assumed that CMC was devoid of any nonverbal cues, the
``cues-filtered-out'' view. This was one of the first studies to notice
that there are nonverbal, chronemic cues in CMC. It found that altering
time and date stamps changed the way external observers saw the
communication. \textcite{walther95} validated two hypotheses and had
complex results others. First, it proved (their H2) that a social
message sent at night signals less dominance than a social message sent
during the day and that a task message sent at night is more dominant
than a task message sent during the day. This agrees with their
assertion that ``For instance, different parts of the day correspond to
different activity contexts, and formal `business hours' versus informal
`after hours' carry different expectations {[}30{]}'' \autocite[
361]{walther95}. Second, it proved (their H3) that a slow reply to a
social message indicates greater affection than a fast reply, but a slow
reply to a task message indicates less affection than a fast reply
\autocite[ 370]{walther95}.

\textbf{\fullcite{doring09}}

Cites \textcite{walther95}, I think it's cited by quite a few papers
below as well. A little hard to keep track of because it's inside a
book.

\textbf{\fullcite{tyler03}}

Cited by \textcite{kalman11} below, but doesn't cite Walther. Talks
about the expectation of a response by a certain time. Extremely small
sample size and uses interviews.

\textbf{\fullcite{kalman11}}

Cites \textcite{walther95}. About recruiting/job candidates.

\textbf{\fullcite{tu18}}

Cites \textcite{walther95} a couple times, similar stuff covered. About
dating relationships.

\textbf{\fullcite{kalman13}}

Really interesting!! Cites \textcite{walther95}. About generalized
chronemics and the causal relationship between personality and pacing of
conversations.

\textbf{\fulcite{heston17}}

\hypertarget{purdue-library-wishlist}{%
\subsection{Purdue Library wishlist}\label{purdue-library-wishlist}}

\fullcite{ling11}

For \textcite{doring09} mostly.

\printbibliography

\end{document}
