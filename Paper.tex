% Options for packages loaded elsewhere
\PassOptionsToPackage{unicode}{hyperref}
\PassOptionsToPackage{hyphens}{url}
%
\documentclass[
  stu]{apa7}
%
\PassOptionsToPackage{style=apa}{biblatex}
%\PassOptionsToPackage{indent=7ex}{parskip}
%\usepackage{indentfirst}
%\usepackage[margin=1in]{geometry}
%% right side page number
%\usepackage{fancyhdr}
%\fancypagestyle{plain}{
%\fancyhf{}
%\renewcommand{\headrulewidth}{0pt}
%\fancyhead[R]{\thepage}
%}
%\pagestyle{fancy}
%\fancyhf{}
%\renewcommand{\headrulewidth}{0pt}
%\fancyhead[R]{\thepage}
%% double spacing
%\usepackage{setspace}
%\doublespacing
\affiliation{Brian Lamb School of Communication}
\course{COM 212: Interpersonal communication}
\professor{Dr. Hayden Barber}
\duedate{March 31, 2022}
%
\title{Chronemics in CMC}
\author{Sean F. O'Donovan}
\date{}

\usepackage{amsmath,amssymb}
\usepackage{lmodern}
\usepackage{iftex}
\ifPDFTeX
  \usepackage[T1]{fontenc}
  \usepackage[utf8]{inputenc}
  \usepackage{textcomp} % provide euro and other symbols
\else % if luatex or xetex
  \usepackage{unicode-math}
  \defaultfontfeatures{Scale=MatchLowercase}
  \defaultfontfeatures[\rmfamily]{Ligatures=TeX,Scale=1}
\fi
% Use upquote if available, for straight quotes in verbatim environments
\IfFileExists{upquote.sty}{\usepackage{upquote}}{}
\IfFileExists{microtype.sty}{% use microtype if available
  \usepackage[]{microtype}
  \UseMicrotypeSet[protrusion]{basicmath} % disable protrusion for tt fonts
}{}
\makeatletter
\@ifundefined{KOMAClassName}{% if non-KOMA class
  \IfFileExists{parskip.sty}{%
    \usepackage{parskip}
  }{% else
    \setlength{\parindent}{0pt}
    \setlength{\parskip}{6pt plus 2pt minus 1pt}}
}{% if KOMA class
  \KOMAoptions{parskip=half}}
\makeatother
\usepackage{xcolor}
\IfFileExists{xurl.sty}{\usepackage{xurl}}{} % add URL line breaks if available
\IfFileExists{bookmark.sty}{\usepackage{bookmark}}{\usepackage{hyperref}}
\hypersetup{
  pdftitle={Chronemics in CMC},
  pdfauthor={Sean F. O'Donovan},
  hidelinks,
  pdfcreator={LaTeX via pandoc}}
\urlstyle{same} % disable monospaced font for URLs
\setlength{\emergencystretch}{3em} % prevent overfull lines
\providecommand{\tightlist}{%
  \setlength{\itemsep}{0pt}\setlength{\parskip}{0pt}}
\setcounter{secnumdepth}{-\maxdimen} % remove section numbering
\ifLuaTeX
  \usepackage{selnolig}  % disable illegal ligatures
\fi
\usepackage[]{biblatex}
\addbibresource{comPaper1References.bib}

\begin{document}
\maketitle

\hypertarget{introduction}{%
\section{Introduction}\label{introduction}}

When I communicate with others using computer mediated communication
(instant messaging, sms, social networks, also known as cmc), I
frequently notice that the conversation has a rhythm and pace. For
example, responding quickly to messages says something different from
responding slowly. Another example is that texting someone at 6 in the
morning feels very different from texting them at midnight.

In interpersonal communication, this is called chronemics. The
phenomenon I am choosing to investigate in this paper is chronemics in
computer mediated communication. Given that people today are
communicating with each other more than ever using social media and
instant messaging, I think this topic is important to understand.

To define this cmc chronemics clearly, I mean the frequency of
communication, the pacing of the conversation itself (which is sometimes
the same as the frequency of communication), and the time at which it
takes place. My personal belief as I explore this topic is that
chronemics communicate a lot in computer mediated communication, and to
some extent take the place of nonverbal cues. Additionally, I think
chronemics can be used to pull empirical data about relationships out of
individual and aggregated conversations. This paper will begin by
examining the way researchers have studied the existence of chronemics
in cmc, and then discuss the ways researchers have been able to put
chronemics to use. Researchers began to document the existence of
chronemics in cmc in the 90s, and are still catching up to the pace of
communication today.

\hypertarget{useful-theories}{%
\section{Useful Theories}\label{useful-theories}}

To start examining the literature on computer-mediated chronemics, we
need to define and explain some general theories. The first and
traditional theory is cues-filtered-out theory.

This theory was introduced by \textcite{sproull91}, and is an extremely
naive approach. It argues that cmc and cyberspace exist without any
non-verbal cues, and therefore is a fundamentally different space. The
main conclusion is that there will be fewer and less intense
relationships in cyberspace. A side effect, according to the theory, is
that relationships are less hierarchical and more democratic; everyone
can have an equal share of the conversation. A good summary of the
theory is that it views cmc as ``inherently impersonal.'' The
cues-filtered-out approach is the traditional approach that most other
theories exist to disprove, and I think most people who have experienced
cyberspace in any meaningful way would agree that this theory is
seriously lacking.

The second theory to talk about is the social information processing
(SIP) perspective. It claims that people are able to edit how they
present themselves more, and people give each other the benefit of the
doubt more in cyberspace. This means that there is an opportunity for
even more emotion and intensity of relationship than normal in
cyberspace. It also claims that because people are missing normal cues,
they fixate on what cues there are, such as writing style, written
``nonverbals'' like emojis, and other text cues like repeated letters.

\hypertarget{describing-cmc-chronemics-phenomena}{%
\section{Describing CMC Chronemics
Phenomena}\label{describing-cmc-chronemics-phenomena}}

The first and most important discovery by researchers about cmc
chronemics was the acknowledgement that they exist and influence
people's impression of the conversation. This is the claim made by
\textcite{walther95}, in contrast to the previous view that cmc is
impersonal and doesn't contain nonverbal cues. This was the first, most
important paper which refuted cues-filtered-out theory.

This was effectively replicated for sms by \textcite{doring09}.

The next important phenomenon documented by researchers is that of
online silence. The idea that even in asynchronous communications, it is
possible for silence to exist was documented by \textcite{ravid} and
also by \textcite{kalman05}. Kalman explains that ``silence can be
defined as no response after an x period of time, at which, say, 99\% or
97\% of the responses have already been created'' (2005).
\textcite{kalman05} also explains that the silence generated there can
have major disruptive effects on online communication, from interfering
with team collaboration, to creating misunderstandings.

This can be viewed through the lens of expectation violation theory as
well, as described in \textcite{kalman11} and \textcite{sheldon06}.
Surprisingly, they find different results; both concluded that the
reward valence of the person violating chronemic norms changes how that
violation is perceived, but \textcite{sheldon06} found that low-reward
violaters were more simply more negatively perceived that high-reward
violaters. \textcite{kalman11} found a more complex interaction,
although both studies agreed that the norm violation was perceived
negatively. Part of that is the context, as both the studies were
centered around work. That lines up with Walther's claims that task
focused messages with high latencies were percieved negatively. These
studies validate SIP theory by suggesting that chronemics are a form of
nonverbal communication that participants tend to read into; cues
filtered out theory would expect participants to ignore or discount the
chronemics because the communication is happening in cyberspace.

\hypertarget{chronemics-applications}{%
\section{Chronemics Applications}\label{chronemics-applications}}

Part of the reason cmc chronemics are so interesting is that people
build the systems that these interactions occur in; to me that means
that if we can discover patterns in cmc chronemics that are well
understood, people can apply those patterns when building cmc systems.
For example, if a short response latency automatically meant that two
communicators were closer to each other, that would be useful for
measuring relationship strength and recommending future relational links
in a social network. As far as I can tell, no such simple application
exists. Instead, it seems that chronemics are being analyzed, but there
isn't all that much interest in it as a measurement. Part of this is
because of contamination by other correlated variables, which will be
discussed later. First, there are some promising uses of chronemics,
starting with an apparent correlation with personality type discovered
by \textcite{kalman13}.

This study evaluates a measure called interpost pause. It builds on the
findings of \textcite{kalman11} and SIP theory by using chronemics as
nonverbal cues that participants use in place of normal face to face
cues. \textcite{kalman13} found that people who were more extraverted
``exhibited shorter interpost pauses,'' and that pairs who trusted each
other less had longer interpost pauses (section 4.1 para 1).
Interestingly, the correlation with trust was stronger than the
correlation with extraversion. Unfortunately, it's hard to know why
these two results exist. For trust, \textcite{kalman13} suggests it may
come down to the assumption that lying is harder than telling the truth
(and therefore should take longer), but it may also be that people just
prefer their conversation partner to respond quickly and the dislike of
slow responses bleeds into their trust for the person. For extraversion,
\textcite{kalman13} suggests this is mostly because extraverted people
talk more and with less hesitation than intraverted people. Essentially,
the same thing happens in face to face communication because that's just
the way people are. To me, this is incredible because it opens the
possibility of studying those attributes with access to chat logs and
without surveys which can be difficult to sample properly. For example,
if Facebook wants to find out which of its users trust each other, it
could conceivably make a computer model depending on interpost pause in
messaging to find that out. Previously, they would have had to do a
survey, then write a model comparing attributes in the survey to user
attributes. Those attributes may not have yielded any correlation at
all.

The idea that interpost pause or measurable chronemics of any kind could
be very useful is something that for some reason, I couldn't find many
studies on. In particular, I was surprised I didn't see studies using it
to weight relationships and generate tie strength measurements. Facebook
and other social medias generally recommend ``friends'' or other social
links algorithmically, and in order to do that, they usually need
measures of a user's current relationships and their strengths. Link
recommender systems for social media are incredibly important and
profitable, and if you can generate better measures of relationships,
you should get a better recommendation. First, we'll discuss some cases
where chonemics were able to improve tie strength prediction.

\textcite{arnaboldi13} is a good example of this. In particular, they
use many, many chronemic variables. The most effective were ``number of
days since last communication, the frequency of contact (bidirectional
and related to incoming interactions) and the number of days since first
communication'' \autocite[ 1137]{arnaboldi13}. Recency of communication
was a great predictor of tie strength, validating the idea that tie
strength should be predicted by chronemics. Unfortunately, most
researchers (like \textcite{servia-rodriguez14}) only look at recency
and duration of relationship. While this makes sense from a running time
standpoint, it seems that interpost pause or time frame of most
communication (like from \textcite{walther95}) could be useful.

Here we'll discuss studies by \textcite{marsden84} and
\textcite{liberatore17} that may provide an answer to why chronemics
isn't used more. Tie strength measurements in social networks (in
general rather than the cmc specific meaning) have existed for a long
time. One study which provides an example of that is
\textcite{marsden84}. It considers multiple variables for constructing
tie strength in social networks. In particular, they consider
measurements for duration and frequency of contact, in face to face
interaction. The most important part of the study is that they find
issues with using duration and frequency of contact, because there are
so many confounding variables with those two measures. One example they
discuss is neighbors - you may see your neighbor every day, and be
dragged into an hour long conversation with them every day, and still
dislike them or feel indifferent towards them. I think this type of
confounding tends to make researchers apprehensive when they consider
attempting studies like Kalman et al's 2013 study. There are other
explanations though.

One can be found in \textcite{liberatore17}, in which a meta-study is
conducted on the different ways to generate tie strength in cmc social
networks. This is the measurement that I would expect to see chronemics
factor into. The study does consider chronemics with one measure:
Duration (of relationship). Unfortunately, many of the approaches don't
really consider chronemics outside of that factor. One of the reasons
may be the difficulty of generating the data for every user; the studies
which did include chronemics of individual pair conversations typically
used a lot of variables, and still didn't consider interpost pause.
Unfortunately, the more variables used to generate tie strength, the
worse the model can be to use, because the model can risk over fitting
the data and being quite slow and expensive to use. In particular, if
Facebook wants to generate tie strengths for their entire network, it
will prefer models that are simpler because they take less computation
time and are more reliable.

That said, it's pretty odd to ignore nonverbal cues when researchers try
to predict the strength of a relationship in cmc. When meeting a pair of
people in real life, some of the first things anyone will look at to
understand their relationship are the nonverbal cues in the
conversation. Obviously in real life we can just ask, but Facebook won't
ask every user to rate every friend in their account.

\hypertarget{conclusion}{%
\section{Conclusion}\label{conclusion}}

To look ahead a bit, this research suggests that there is an opportunity
to try and predict tie strength with interpost pause and aggregate time
of messaging. There is also a possibility that this has been done
already, but that the research isn't published. In any event, cmc
definitely has influential nonverbal cues in the form of chronemics.

Researchers have refuted theories that cmc doesn't have chronemic
nonverbal cues. They have documented their existence and influence on
conversations, and have used them to some extent, although there is
still much to be exploited in terms of their use.

\printbibliography

\end{document}
