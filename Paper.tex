% Options for packages loaded elsewhere
\PassOptionsToPackage{unicode}{hyperref}
\PassOptionsToPackage{hyphens}{url}
%
\documentclass[
  stu]{apa7}
%
%\PassOptionsToPackage{style=apa}{biblatex}
%\PassOptionsToPackage{indent=7ex}{parskip}
%\usepackage{indentfirst}
%\usepackage[margin=1in]{geometry}
%% right side page number
%\usepackage{fancyhdr}
%\fancypagestyle{plain}{
%\fancyhf{}
%\renewcommand{\headrulewidth}{0pt}
%\fancyhead[R]{\thepage}
%}
%\pagestyle{fancy}
%\fancyhf{}
%\renewcommand{\headrulewidth}{0pt}
%\fancyhead[R]{\thepage}
%% double spacing
%\usepackage{setspace}
%\doublespacing
\affiliation{Brian Lamb School of Communication}
\course{COM 212: Interpersonal communication}
\professor{Dr. Hayden Barber}
\duedate{March 31, 2022}
\PassOptionsToPackage{style=apa}{biblatex}
%
\title{Chronemics in CMC}
\author{Sean F. O'Donovan}
\date{}

\usepackage{amsmath,amssymb}
\usepackage{lmodern}
\usepackage{iftex}
\ifPDFTeX
  \usepackage[T1]{fontenc}
  \usepackage[utf8]{inputenc}
  \usepackage{textcomp} % provide euro and other symbols
\else % if luatex or xetex
  \usepackage{unicode-math}
  \defaultfontfeatures{Scale=MatchLowercase}
  \defaultfontfeatures[\rmfamily]{Ligatures=TeX,Scale=1}
\fi
% Use upquote if available, for straight quotes in verbatim environments
\IfFileExists{upquote.sty}{\usepackage{upquote}}{}
\IfFileExists{microtype.sty}{% use microtype if available
  \usepackage[]{microtype}
  \UseMicrotypeSet[protrusion]{basicmath} % disable protrusion for tt fonts
}{}
\makeatletter
\@ifundefined{KOMAClassName}{% if non-KOMA class
  \IfFileExists{parskip.sty}{%
    \usepackage{parskip}
  }{% else
    \setlength{\parindent}{0pt}
    \setlength{\parskip}{6pt plus 2pt minus 1pt}}
}{% if KOMA class
  \KOMAoptions{parskip=half}}
\makeatother
\usepackage{xcolor}
\IfFileExists{xurl.sty}{\usepackage{xurl}}{} % add URL line breaks if available
\IfFileExists{bookmark.sty}{\usepackage{bookmark}}{\usepackage{hyperref}}
\hypersetup{
  pdftitle={Chronemics in CMC},
  pdfauthor={Sean F. O'Donovan},
  hidelinks,
  pdfcreator={LaTeX via pandoc}}
\urlstyle{same} % disable monospaced font for URLs
\setlength{\emergencystretch}{3em} % prevent overfull lines
\providecommand{\tightlist}{%
  \setlength{\itemsep}{0pt}\setlength{\parskip}{0pt}}
\setcounter{secnumdepth}{-\maxdimen} % remove section numbering
\ifLuaTeX
  \usepackage{selnolig}  % disable illegal ligatures
\fi
\usepackage[]{biblatex}
\addbibresource{comPaper1References.bib}

\begin{document}
\maketitle

\hypertarget{introduction}{%
\section{Introduction}\label{introduction}}

When I communicate with others using computer mediated communication
(instant messaging, sms, social networks, also known as cmc), I
frequently notice that the conversation has a rhythm and pace. For
example, responding quickly to messages says something different from
responding slowly. Another example is that texting someone at 6 in the
morning feels very different from texting them at midnight.

In interpersonal communication, this is called chronemics. The
phenomenon I am choosing to investigate in this paper is chronemics in
computer mediated communication. Given that people today are
communicating with each other more than ever using social media and
instant messaging, I think this topic is important to understand.

To define this cmc chronemics clearly, I mean the frequency of
communication, the pacing of the conversation itself (which is sometimes
the same as the frequency of communication), and the time at which it
takes place. My personal belief as I explore this topic is that
chronemics communicate a lot in computer mediated communication, and to
some extent take the place of nonverbal cues. Additionally, I think
chronemics can be used to pull empirical data about relationships out of
individual and aggregated conversations. This paper will begin by
examining the way researchers have studied the existence of chronemics
in cmc, and then discuss the ways researchers have been able to put
chronemics to use.

\hypertarget{useful-theories}{%
\section{Useful Theories}\label{useful-theories}}

To start examining the literature on computer-mediated chronemics, we
need to define and explain some general theories. The first and
traditional theory is cues-filtered-out theory.

This theory was introduced by \textcite{sproull91}, and is an extremely
naive approach. It argues that cmc and cyberspace exist without any
non-verbal cues, and therefore is a fundamentally different space. The
main conclusion is that there will be fewer and less intense
relationships in cyberspace. A side effect, according to the theory, is
that relationships are less hierarchical and more democratic; everyone
can have an equal share of the conversation. A good summary of the
theory is that it views cmc as ``inherently impersonal.'' The
cues-filtered-out approach is the traditional approach that most other
theories exist to disprove, and I think most people who have experienced
cyberspace in any meaningful way would agree that this theory is
seriously lacking.

The second theory to talk about is the social information processing
perspective. It claims that people are able to edit how they present
themselves more, and people give each other the benefit of the doubt
more in cyberspace. This means that there is an opportunity for even
more emotion and intensity of relationship than normal in cyberspace. It
also claims that because people are missing normal cues, they fixate on
what cues there are, such as writing style, written ``nonverbals'' like
emojis, and other text cues like repeated letters.

The next important theory to be aware of is social presence theory.
Online social presence in this context is the appearance or perception
of a participant being a real person or being present in the
conversation in cmc \autocite{cui13}. It's important to this paper
because it is influenced by chronemics.

\hypertarget{describing-cmc-chronemics-phenomena}{%
\section{Describing CMC Chronemics
Phenomena}\label{describing-cmc-chronemics-phenomena}}

The first and most important discovery by researchers about cmc
chronemics was the acknowledgement that they exist and influence
people's impression of the conversation. This is the claim made by
\textcite{walther95}, in contrast to the previous view that cmc is
impersonal and doesn't contain nonverbal cues.

This was effectively replicated for sms by \textcite{doring09}.

The next important phenomenon documented by researchers is that of
online silence. The idea that even in asynchronous communications, it is
possible for silence to exist was documented by \textcite{ravid} and
also by \textcite{kalman05}. Kalman explains that ``silence can be
defined as no response after an x period of time, at which, say, 99\% or
97\% of the responses have already been created'' (2005).
\textcite{kalman05} also explains that the silence generated there can
have major disruptive effects on online communication, from interfering
with team collaboration, to creating misunderstandings.

This can be viewed through the lens of expectation violation theory as
well, as described in \textcite{kalman11} and \textcite{sheldon06}.
Surprisingly, they find different results; both concluded that the
reward valence of the person violating chronemic norms changes how that
violation is perceived, but \textcite{sheldon06} found that low-reward
violaters were more simply more negatively perceived that high-reward
violaters. \textcite{kalman11} found a more complex interaction,
although both studies agreed that the norm violation was perceived
negatively. Part of that is the context, as both the studies were
centered around work. That lines up with Walther's claims that task
focused messages with high latencies were percieved negatively.

\hypertarget{chronemics-applications}{%
\section{Chronemics Applications}\label{chronemics-applications}}

Part of the reason cmc chronemics are so interesting is that people
build the systems that these interactions occur in; to me that means
that if we can discover patterns in cmc chronemics that are well
understood, people can apply those patterns when building cmc systems.
For example, if a short response latency automatically meant that two
communicators were closer to each other, that would be useful for
measuring relationship strength and recommending future relational links
in a social network. As far as I can tell, no such simple application
exists. Instead, it seems that chronemics are being analyzed, but there
isn't all that much interest in it as a measurement. Part of this is
because of contamination by other correlated variables, which will be
discussed later. First, there are some promising uses of chronemics,
starting with an apparent correlation with personality type discovered
by \textcite{kalman13}.

\printbibliography[title=Conclusion]

\end{document}
